\documentclass[]{letter}
\usepackage{amsmath}
\usepackage{graphicx}
\usepackage{fullpage}
 
\pagestyle{empty} 
\usepackage[margin=0.8in]{geometry}
 
\begin{document}
 
\textbf{\Large{PHSX 331 Homework 5 Due October 3 5:00 pm - Magnetic Zero}}
 
\hrulefill
 
\textbf{\underline{Question 1}} You have a set of parallel wires in your lab, and you want to know where the magnetic field is zero between them. To start, lets look at the magnetic field from just a singular wire with current $I$. In physics 2 (PHSX 222/242) you learned that the magnetic field from a current carrying wire wrapped around the wire following the right hand rule. When you take E\&M you derive that for an infinite wire the magnetic field around it is,
\begin{equation}
	\boldsymbol{B} = \frac{\mu_0 I}{2\pi s}\boldsymbol{\hat{\phi}}
\end{equation}
Where $s$ is the distance from the wire to the point we want to evaluate the magnetic field at. In class we wrote a program that found the magnetic field for a single wire located at the origin.
\begin{itemize}
	\item Modify that program so that our wire can be located at any point. The function should have 3 inputs, two should be arrays, one for where we want to find the value of the magnetic field, and a second array for where the wire is located. The third should be the current in the wire, positive current should be out of the page, and negative current should be into the page.
	\item Ideally the set up of wires would be a square with sides $0.5$cm and each with a current $10$A. Set up your coordinates so that the bottom left corner of the square is at the origin, the current is into the page for the bottom left and top right wires, and the current is out of the page for the top left and bottom right wires, see the picture. Write a function that returns the total magnetic field from each of these wires. 
	\begin{center}
		\includegraphics[scale = 1]{"HW 5 Picture".png}
	\end{center}
	\item Use \textit{fmin} to find the point where the magnetic field is zero. Use your intuition to check the answer. You might have to mess with the \textit{xtol} and \textit{ftol} limits in \textit{fmin} to get the correct answer.
	\item You thought setting up something so simple could be left to one of your friends who only does theory. When you show up at the lab this set up is completely wrong. The wires aren't in a square, and they don't have the same current, see the Non-Ideal figure for position and currents. Where will the magnetic field be zero?
	\begin{center}
		\includegraphics[scale = 1]{"HW 5 Picture 2".png}
	\end{center}
	\item Extra Credit: Make a vector plot for both the ideal and non-ideal wire setups. If the plots look nice I'll give you 2 points of extra credit.\footnote{'nice' is 100\% subject to my discretion.}
\end{itemize} 
 
\end{document}