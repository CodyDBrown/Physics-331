\documentclass[]{letter}
\usepackage{amsmath}
\usepackage{graphicx}
\usepackage{fullpage}
 
\pagestyle{empty} 
\usepackage[margin=0.8in]{geometry}
 
\begin{document}
 
\textbf{\Large{PHSX 331 HW 7. (DATE) - Laser Threshold}}
 
\hrulefill
 
\textbf{\underline{Question 1}} 

 A solid state laser is a collection of special atoms that we'll call "laser active ". These are arranged in a solid state matrix between two reflective mirrors. A source of energy (pump) excites the atoms out of the ground state. These excited atoms will release their energy as a photon, and drop back down to the ground state. If the pump is weak, then the laser will act like a lamp, each atom oscillates independently and emits a random photon of light. Once the pump reaches a certain threshold then the atoms will oscillate in phase and the lamp becomes a laser. The number of photons $n(t)$ has a rate of change given by,
\begin{align}
	 \frac{dn}{dt} &= \text{gain} - \text{loss} \\
		&= GnN - kn
\end{align}

$N(t)$ is the number of exited atoms, $G$ is the gain coefficient, $k$ is a constant so that $1/k$ is the expected lifetime of a photon. When an excited atom emits a photon it drops back down to the ground state, so we can write the number of atoms in excited states can be written as,
\begin{align}
N(t) = N_{0} - \alpha n
\end{align}
$N_{0}$ is the number of excited atoms the pump can keep excited, $\alpha$ is the decay rate, then the differential equation for the number of photons is,
\begin{equation}
	\frac{dn}{dt} = (GN_{0} - k )n - (\alpha G) n^2
\end{equation}
For all parts use $G = 5$, $\alpha = 3$, $k = 30$.
\begin{itemize}
	\item What are the units of $G$, $N_0$, $k$, and $\alpha$?
	\item Write a function in Python for $\frac{dn}{dt}$.
	\item Use rk2\_1d.py to solve for $n(t)$ with an initial condition of $n(0) = 100$. Run the program for pump strength of $N_0 = 1, 10 \text{ and } 100$. 
	\item Plot $n$ as a function of time for each of the different pump strengths. Make sure to label and add a legend to the plot.
	\item Make a second plot for a pump strength of $N_0 = 1$ and an initial condition of $n(0) = 0, \; 10^2, \text{ and } 10^4$ photons. How does the initial condition affect the final state of the laser.
	\item Make a third plot with a pump strength of $N_0 = 10$ with initial conditions $n(0) = 0, \; 10^2, \text{ and } 10^4$ photons. How does the initial condition affect the final state of the laser.
	\item Based on your graphs what matters more in making the laser, the number of photons you start with, of the strength of the pump. 
	\item Try to find the Threshold $N_0$ where the atoms switch between just being a lamp, and becoming a laser. 
\end{itemize}
\end{document}